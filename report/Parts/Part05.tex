\chapter{Conclusion and Considerations}

\section{OpenGL 3+ on different hardware}

One of the biggest difficulties encountered was to make the program work on
different graphic cards. Luckily enough, the two laptops used to work on the project had different
hardware (one nVidia and one ATI) so it gives stronger chances to detect some problems.
 
For instance, at some point the program worked on both computers, but after an update
of the graphics driver the one with nVidia hardware wouldn't render the scene any more, displaying
silently a black screen instead. That was given by some initialization steps that were
done in the wrong order (some code that was written a week before without us noticing it),
and some texture initialisations were missing, with ATI drivers being more permissive than nVida's on
the matter.

The use of gDebugger helped a lot to understand what was going on inside,
since a traditionnal debugguer cannot inspect what happens within OpenGL and since OpenGL
does not give much information on errors.

gDebugger intercepts draw calls and displays the buffers that are
on gpu memory (textures, render buffers, vertex buffers, etc.), which is convenient
when implementing deferred shading.

gDebugger, however, can cause the disappearance of specific OpenGL related Heisenbugs. This
happened a great number of times. The texture initialisation problem was not debuggable
with gDebugger since the program bypasses some bad initialisations and uses some standard
parameters, whereas in a non controlled environment this does not happen. One of the
greatest mysteries that were encountered during the development of the project that was
resolved in the last days was caused by a wrong initialisation of the textures, once again,
but this time it was due to a removal of support for a specific enumerator. This, however,
did not throw any error but caused some random bugs, such as showing incorrect rendering,
reading the wrong data from the texture during deferred shading and such.

This problem was due to the fact that OpenGL 3+ does not have any more the \syntax{GL\_CLAMP}
option for the \syntax{glTexParameteri} function but instead has two new options:
\syntax{GL\_CLAMP\_TO\_EDGE} and \syntax{GL\_CLAMP\_TO\_BORDER}. The first one was
used in the project and, after replacing it some of the bugs, such as incorrect
edge detection and blurring on the borders or wrong initialisation, disappeared.

Another problem was the support of experimental features on the OpenGL context used
in the project. To be forward compatible the project was written using Vertex
Array Objects to contain informations on the vertices rendered on the screen.
This worked ``Out of the box'' on the ATI card (most probably because it was
more recent), while was not supported on the nVidia one. To fix this, however,
the Glew was used to enable experimental extensions for the OpenGL context used,
by enabling the \syntax{glewExperimental} flag in initialisation.	
